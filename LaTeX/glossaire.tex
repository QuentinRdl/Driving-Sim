\newglossaryentry{dynamique_vehicule}{
    name={Dynamique de véhicule},
    description={Ensemble des lois physiques qui régissent le mouvement, la stabilité et le comportement d’un véhicule.}
}

\newglossaryentry{integration_numerique}{
name={Intégration numérique},
description={Méthode permettant d’approximer la solution des équations différentielles dans les simulations en temps réel.}
}

\newglossaryentry{euler_explicite}{
name={Méthode d'Euler explicite},
description={Technique d’intégration numérique simple utilisée pour mettre à jour l’état d’un système (vitesse, position, etc.).}
}

\newglossaryentry{modele_bicycle}{
name={Modèle Bicycle simplifié},
description={Modèle mathématique regroupant les roues avant et arrière en deux entités distinctes afin de simplifier la simulation de la dynamique du véhicule.}
}

\newglossaryentry{angle_glissement}{
name={Angle de glissement (Slip angle)},
description={Angle formé entre la direction réelle d’un pneu et son orientation, déterminant la force latérale développée.}
}

\newglossaryentry{sous_virage}{
name={Sous-virage},
description={Comportement où le véhicule tourne moins que la trajectoire attendue, généralement dû à une perte d’adhérence à l’avant.}
}

\newglossaryentry{survirage}{
name={Survirage},
description={Comportement où le véhicule tourne plus que prévu, souvent lié à une perte d’adhérence à l’arrière.}
}

\newglossaryentry{centre_gravite}{
name={Centre de gravité},
description={Point de concentration de la masse d’un véhicule, influençant sa stabilité et la répartition des forces.}
}

\newglossaryentry{essieu}{
name={Essieu},
description={Composant reliant les roues d’un véhicule sur le même axe, dont la répartition du poids influe sur la dynamique.}
}

\newglossaryentry{moment_inertie}{
name={Moment d'inertie},
description={Mesure de la résistance d’un corps à un changement de rotation, calculée à partir de la masse et de la répartition des distances par rapport au centre de gravité.}
}

\newglossaryentry{raideur_pneus}{
name={Raideur des pneus},
description={Caractéristique des pneus qui détermine leur réponse aux forces appliquées. Voir aussi \gls{raideur_longitudinale} et \gls{raideur_laterale}.}
}

\newglossaryentry{raideur_longitudinale}{
name={Raideur longitudinale (Cx)},
description={Mesure de la réactivité des pneus aux forces de traction ou de freinage.}
}

\newglossaryentry{raideur_laterale}{
name={Raideur latérale (Cy)},
description={Mesure de la réponse des pneus aux forces générées lors des virages.}
}

\newglossaryentry{glissement_dynamique}{
name={Glissement dynamique},
description={Évolution dans le temps du glissement d’un pneu par rapport à la surface de la route.}
}

\newglossaryentry{saturation_forces_laterales}{
name={Saturation non linéaire des forces latérales},
description={Phénomène limitant la force latérale qu’un pneu peut générer une fois qu’un certain angle de glissement est dépassé.}
}

\newglossaryentry{coefficient_adherence}{
name={Coefficient d'adhérence},
description={Paramètre définissant l’adhérence maximale d’un pneu sur la route, influençant les comportements de sous-virage et survirage.}
}

\newglossaryentry{delta_temps}{
name={Delta de temps (dt)},
description={Intervalle de temps entre deux itérations de la simulation, essentiel pour l’intégration numérique.}
}

\newglossaryentry{gnuplot}{
name={Gnuplot},
description={Outil de traçage graphique utilisé pour visualiser les résultats de la simulation (courbes de vitesse, trajectoire, etc.).}
}

\newglossaryentry{sfml}{
name={SFML (Simple and Fast Multimedia Library)},
description={Bibliothèque C++ facilitant la création d’applications graphiques et multimédia.}
}

\newglossaryentry{hud}{
name={HUD (Head-Up Display)},
description={Système d’affichage superposé sur l’écran, utilisé ici pour présenter des informations telles que le compteur de FPS et les données de débogage.}
}

\newglossaryentry{cmake}{
name={CMake},
description={Outil de configuration et de gestion de compilation des projets C++.}
}

\newglossaryentry{std_deque}{
name={\texttt{std::deque}},
description={Structure de données de la STL C++ utilisée pour gérer une file de deltas de temps.}
}

\newglossaryentry{std_vector}{
name={\texttt{std::vector}},
description={Conteneur dynamique de la STL C++ servant à stocker des données, par exemple, les points de la trajectoire prédictive.}
}

\newglossaryentry{std_nth_element}{
name={\texttt{std::nth\_element}},
description={Algorithme de la STL C++ permettant d’obtenir efficacement l’élément médian dans un conteneur non trié.}
}

\newglossaryentry{std_sort}{
name={\texttt{std::sort}},
description={Algorithme de tri de la STL C++ utilisé pour ordonner les éléments d’un conteneur.}
}

\newglossaryentry{std_unordered_map}{
name={\texttt{std::unordered\_map}},
description={Structure de données de la STL C++ qui associe des clés à des valeurs, utilisée pour la gestion des ressources (textures, etc.).}
}

\newglossaryentry{std_unique_ptr}{
name={\texttt{std::unique\_ptr}},
description={Pointeur intelligent de la STL C++ assurant la gestion automatique de la mémoire pour des objets, par exemple, la police de caractères.}
}
