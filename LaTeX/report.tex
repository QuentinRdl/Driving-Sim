\documentclass[a4paper,12pt]{report}
\usepackage[utf8]{inputenc}
\usepackage[T1]{fontenc}
\usepackage{graphicx}
\usepackage{geometry}
\usepackage{setspace}
\usepackage{titling}
\usepackage{fancyhdr}
\usepackage{ifthen}
\usepackage{lastpage}
\usepackage{xurl}  % Permet de casser les URL de manière améliorée
\usepackage[breaklinks]{hyperref}  % Permet de casser les liens URL
\def\UrlBreaks{\do\/\do-}
\usepackage[french]{babel}
\renewcommand\thesection{\Roman{section}} % Numérotation des sections en chiffres romains
\renewcommand\thesubsection{\arabic{subsection}} % Numérotation des sous-sections en chiffres arabes

% Pour gerer l'espace dans la table of contents
\usepackage{tocloft}
\renewcommand{\cftsecnumwidth}{2.3em} % Ajustez la largeur en fonction de vos besoins
\renewcommand{\cftsubsecnumwidth}{2.3em} % Ajustez la largeur en fonction de vos besoins
% ---

\hypersetup{
    colorlinks=true,
    linkcolor=black, % Couleur des liens internes (table des matières, etc.)
    urlcolor=blue, % Couleur des liens externes (URLs)
    citecolor=blue, % Couleur des liens de citation
    filecolor=blue % Couleur des liens vers des fichiers
}

\fancyhf{} % Efface les en-têtes et pieds de page par défaut
\pagestyle{fancy}
\renewcommand\headrulewidth{1pt}
\fancyhead[L]{Romain GALLAND | Quentin RADLO}
\fancyhead[R]{Université de Marie \& Louis Pasteur}
\renewcommand\footrulewidth{1pt}
\fancyfoot[L]{}
\fancyfoot[C]{Rapport de Projet - Simulateur de conduite:\\
\textbf{Page \thepage/\pageref{LastPage}}}
\fancyfoot[R]{}
\geometry{
  bindingoffset=1cm,  % Décalage pour la reliure
  left=2cm,
  right=2cm,
  top=2.5cm,
  bottom=2.5cm
}

\setlength{\headheight}{14.5pt}
\addtolength{\topmargin}{-2.5pt}

\begin{document}

\begin{titlepage}

 \begin{center}
    \vspace{2cm}
    \includegraphics[width=0.3\textwidth]{university_logo.png}\par
  \end{center}



  \begin{center}
    \vspace{0.2cm}
    {\scshape \large{Rapport de projet} \par}
    Licence 3 Informatique - UFR Sciences et Techniques
    \vspace{0.4cm}
    \hrule
    \vspace{0.4cm}
    {\huge\bfseries Simulateur de conduite \par}
    \vspace{0.4cm}
    \hrule
    \vspace{1cm}
    {\large\bfseries Décembre 2024 - Mars 2025 \par}
    Romain GALLAND - Quentin RADLO
    %\vspace*{1.5cm}
 % Image d'illustration
    % \vspace*{1.5cm}
    \vspace{1cm} % Fonctionne mieux si encapsulé dans un bloc centre
    \begin{center}
        \includegraphics[width=0.8\textwidth]{illustration.jpg}
    \end{center}


    % Informations supplémentaires
    \vspace{1cm}
    {\bfseries Sous la supervision de:} Jean-Michel Hufflen \\


  \end{center}
\end{titlepage}


\newpage
\section*{Remerciements}
- Un remerciement particulier est adressé à M. Hufflen pour la guidance et la pédagogie qu'il nous a apporté lors de la réalisation de notre projet.

% Page dédiée à la table des matières
\newpage
\tableofcontents
\newpage


\newpage
\section*{Glossaire}

\textbf{APEC :} Agence pour l'Emploi des Cadres, une association française qui propose des offres d'emploi pour les cadres.

\textbf{Bluetooth :} Norme de communication sans fil à courte portée.

\textbf{C, C++ :} Langages de programmation populaire dans le milieu embarqué.

\textbf{ESP32 :} Microcontrôleur très utilisé dans le développement de projets embarqués.

\textbf{Framework :} Ensemble d'outils et de conventions facilitant le développement.

\textbf{Git :} Système de gestion de versions décentralisé.

\textbf{I2C :} Protocole de communication entre plusieurs dispositifs.

\textbf{LinkedIn :} Réseau social professionnel en ligne.

\textbf{OS :} Système d'exploitation (\textit{Operating System} en anglais).

\textbf{RTOS :} Système d'exploitation temps réel (\textit{Real-Time Operating System} en anglais).

\textbf{SPI :} Protocole de communication série synchrone utilisé principalement pour la communication entre microcontrôleurs et périphériques.

\textbf{SS2I :} Société de Services en Ingénierie Informatique.

\newpage
\section{Sujet du projet}

\subsection{Le sujet}
- Le sujet originel nous laissait une assez grande liberté quant à l'orientation de notre projet. L'idée globale consistait en la réalisation graphique d'un mini-simulateur de conduite, avec des ordres de conduite (démarrage, arrêt, accélération et décélération) donnés par la frappe du clavier. L'utilisateur devait pouvoir observer son véhicule ainsi que le circuit avec une vue du dessus ou la vision du conducteur à travers le pare-brise. Il était aussi proposé de pouvoir "corser" le jeu en programmant des événements aléatoires.

Pour réaliser ce projet, plusieurs langages de programmation nous étaient proposés :
\begin{itemize}
    \item \textbf{DrRacket}, avec des compléments pour nous aider sur la partie graphique.
    \item \textbf{JRuby}, avec pour idée d'implémenter la logique du programme en Ruby et la partie graphique avec des bibliothèques de Java.
    \item \textbf{C\#} ou \textbf{C++} avec les bibliothèques graphiques adéquates.
\end{itemize}


\subsection{Notre interprétation du sujet / Objectif du sujet }
- Après réflexion et discussion avec notre tuteur nous avons décidé de partir vers une orientation didacticiel plutôt qu'une "orientation de 'jeu' ". L'idée était d'implémenter un modèle de dynamique de véhicule se rapprochant de la réalité. Avec un tel modèle nous pourrions donc visualiser des comportements de perte d'adhérence sur la route, ainsi que les phénomènes de survirage et sous virage. Un tel simulateur permettrait donc à l'utilisateur d'expérimenter / se familiariser avec de tel comportements de véhicule. De ce qui est du langage de programmation nous avons décidé de nous orienter vers le langage C++, car il nous était plus familier que les autres langages proposés, et nous avons utilisé SFML en tant que bibliothèque graphique.
- Ensuite nous avons donc commencer l'implémentation de notre simulateur, il s'est décomposé en deux grandes parties, l'implémentation de la physique, et la création de l'interface graphique. Nous allons commencer par vous présenter la partie physique de notre projet.

\section{Domaine d'Activité et Fonction du Développeur Logiciels Embarqués}

\subsection{Mission et Responsabilités}

Le rôle du Développeur Logiciels Embarqués est aussi varié que crucial dans la conception et la mise en œuvre de systèmes embarqués. En tant que professionnel dans ce domaine, le développeur est chargé de créer des logiciels spécifiquement adaptés aux contraintes des dispositifs matériels sur lesquels ils s'exécutent. Cette mission exige une compréhension approfondie du matériel, ainsi que des compétences techniques pointues pour optimiser les performances, la consommation d'énergie et la stabilité des systèmes.

\subsection{Environnement de Travail}

Le Développeur Logiciels Embarqués évolue dans un environnement professionnel stimulant et dynamique, où la convergence entre logiciel et matériel crée une synergie incontournable. Ce professionnel est fréquemment impliqué dans des projets variés, allant de dispositifs électroniques grand public à des applications industrielles critiques, chacun présentant des défis uniques.

L
\newpage

\section{Conclusion}

En résumé, ce dossier confirme ma conviction profonde que le poste de Développeur Logiciels Embarqués est la voie professionnelle qui me correspond parfaitement. L'analyse approfondie des profils recherchés a identifié des compétences techniques essentielles que je m'engage à développer, notamment la maîtrise du langage C/C++, des protocoles de communication et une familiarité avec les RTOS. Mes qualités personnelles telles que l'autonomie, la rigueur et la maîtrise de l'anglais correspondent aux attentes du marché, renforçant ainsi ma conviction d'être un candidat idéal.

L'entretien enrichissant avec M. Abdallah Ben-Othman a non seulement confirmé ma compréhension des missions et responsabilités inhérentes au métier de développeur logiciel embarqué, mais a également renforcé ma conviction que ce poste constitue un défi stimulant et aligné sur mes aspirations professionnelles.

Je suis convaincu que ma solide base de compétences actuelles, combinée à mon engagement à acquérir les connaissances nécessaires, me positionne favorablement pour exceller en tant que développeur logiciel embarqué. Je suis enthousiaste à l'idée de contribuer de manière significative à ce domaine exigeant tout en continuant à cultiver mes compétences dans ce secteur dynamique.

\begin{thebibliography}{99}
  \bibitem{interview}
    Abdallah Ben Othman,
    Interview réalisée le 10 Décembre 2023.

  \bibitem{CreditPhoto}
    \emph{Crédit photo :}
    \url{https://www.it-connect.fr/plus-dun-million-de-raspberry-pi-commercialises/},
    Florian Burnel, sous licence BY-NC-ND 4.0

  \bibitem{APEC}
    \emph{APEC - Offres d'emploi},
    \url{https://www.apec.fr/candidat.html},
    Consulté le 5 Décembre 2023.

  \bibitem{apec}
    \emph{APEC - Fiche Métier},
    \url{https://www.apec.fr/tous-nos-metiers/informatique/ingenieur-en-etudes-et-developpement-informatiques.html},
    Consulté le 19 novembre 2023.

  \bibitem{qt}
    \emph{QT},
    \url{https://www.qt.io/embedded-development-talk/embedded-engineers-roles-responsibilities-and-job-descriptions},
    Consulté le 2 Décembre 2023.

  \bibitem{kicklox}
    \emph{Kicklox},
    \url{https://www.kicklox.com/blog-talent/developpeur-logiciel-embarque},
    Consulté le 2 Décembre 2023.

  \bibitem{yuhiro}
    \emph{Yuhiro},
    \url{https://www.software-developer-india.com/fr/developpeur-de-logiciels-embarques-que-fait-il/},
    Consulté le 2 Décembre 2023.

  \bibitem{ti}
    \emph{Texas Instruments\\},
    \url{https://news.ti.com/blog/2023/04/07/3-trends-impacting-future-embedded-processing-technology},
    Consulté le 12 décembre 2023.

  \bibitem{readwrite}
    \emph{ReadWrite},
    \url{https://readwrite.com/embedded-systems-and-the-future/},
    Consulté le 12 décembre 2023.

\end{thebibliography}
\end{document}