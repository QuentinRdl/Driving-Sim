\section{Bilan Technique}\label{sec:bilan-technique}

\subsection{Problèmes rencontrés et surmontés}\label{subsec:problemes-rencontres-et-surmontes}
Durant le développement du simulateur de conduite, nous avons rencontré et surmonté plusieurs difficultés, ce qui nous a permis d'approfondir nos compétences techniques et d'améliorer notre collaboration.

\subsubsection{Gestion de Git}\label{subsubsec:git}
Tout au long du projet, nous avons utilisé \gls{git} et travaillé sur plusieurs \gls{git_branches} afin d'éviter les conflits de code lors du travail simultané sur les mêmes fichiers.
Malgré ces précautions, quelques conflits sont survenus lors des fusions de branches.
Avec le temps, nous avons appris à mieux gérer cette fonctionnalité de \gls{git} et à améliorer notre communication, réduisant ainsi significativement ces problèmes.

\subsubsection{Configuration avec \gls{cmake}}\label{subsubsec:cmake}
L'utilisation du fichier \texttt{CMakeLists.txt} pour gérer les bibliothèques et créer les différents exécutables a représenté un défi technique supplémentaire.
Les difficultés sont principalement survenues en raison des différences entre nos environnements de développement.
Bien que nous travaillions tous les deux sous Linux, l'un utilisait \gls{arch-linux} et l'autre \gls{ubuntu}, ce qui a engendré quelques disparités dans la gestion de certaines bibliothèques.

\subsubsection{Découverte de \gls{sfml}}\label{subsubsec:sfml}
Nous avons dû nous familiariser avec la bibliothèque \gls{sfml}, alors inconnue pour nous, et concevoir une interface graphique adaptée aux besoins du projet.
Cette phase d'apprentissage a été déterminante pour exploiter pleinement les fonctionnalités de \gls{sfml} et réaliser un rendu graphique efficace sans recourir à des techniques trop complexes.

\subsubsection{Gestion du temps et synchronisation}\label{subsubsec:la-gestion-du-temps}
La synchronisation des mises à jour, notamment la gestion du delta de temps entre les frames, a constitué un défi important.
Il a fallu ajuster nos méthodes de calcul du temps afin d’assurer une fluidité de l’animation et une réactivité optimale de l’interface.

\subsection{Retour sur les technologies utilisées}\label{subsec:retour-sur-les-technologies-utilisees}
Le choix du langage \gls{cpp} s'est révélé judicieux en raison de sa polyvalence et de ses performances.
Malgré une connaissance initiale limitée, ce projet nous a permis de progresser significativement dans sa maîtrise.
L'utilisation de \gls{sfml} a également été un atout majeur, simplifiant le développement d'interfaces graphiques interactives sans nécessiter de techniques trop complexes.

\subsection{Perspectives d'amélioration}\label{subsec:perspectives-d'ameliorations}
Pour les futures itérations du projet, plusieurs axes d'amélioration ont été identifiés :
\begin{itemize}
    \item Enrichir la diversité des circuits proposés afin d'offrir une expérience de conduite plus variée et immersive.
    \item Améliorer l'esthétique globale du projet, tant au niveau de l'interface graphique que des animations.
    \item Affiner le modèle physique en intégrant des phénomènes supplémentaires et en ajustant les paramètres pour une simulation encore plus réaliste.
\end{itemize}

\subsection{Ce que nous ferions différemment}\label{subsec:ce-que-nous-ferions-differemment}
Avec le recul, nous aurions intégré dès le départ des tests unitaires et d'intégration afin de détecter plus rapidement les erreurs et d'améliorer la robustesse du code.
Une phase de planification et de prototypage plus approfondie aurait permis de réduire les imprévus liés à l'apprentissage de nouvelles bibliothèques, notamment \gls{sfml}\@.
Ce changement méthodologique aurait sans doute accéléré le développement et renforcé la qualité du produit final.

\subsection{Mise en route du projet}\label{subsec:mise-en-route-du-projet}

\subsubsection{Dépendances}\label{subsubsec:dependances}
Pour compiler le projet, il est nécessaire d'installer les dépendances suivantes :
\begin{itemize}
    \item \textbf{Un compilateur supportant \gls{cpp}17} : par exemple, \texttt{clang++} ou \texttt{g++}.
    \item \texttt{\gls{cmake}} (version 3.30 ou supérieure) : pour la configuration et la gestion de la compilation.
    \item \texttt{\gls{sfml}} (version 2.5) : pour la gestion de l'interface graphique et des entrées.
    \item \texttt{\gls{gnuplot}} : pour la génération des graphiques.
    \item \texttt{\gls{boost}} : utilisé pour la manipulation des flux, notamment en lien avec \texttt{\glspl{gnuplot}}.
    \item \texttt{\gls{googletest}} : installé automatiquement via le fichier \texttt{CMakeLists.txt}, pour la réalisation des tests unitaires.
\end{itemize}

\subsubsection{Installation}\label{subsubsec:installation}
Le projet est disponible sur \gls{github} à l'adresse suivante : \url{https://github.com/QuentinRdl/Driving-Sim/}.

Pour l'installer, il est nécessaire de cloner le dépôt, et de construire les executables correspondants aux différentes parties du projet grâce à ces commandes :

\begin{lstlisting}[style=bashStyle,label={lst:build}]
git clone 'https://github.com/QuentinRdl/Driving-Sim.git'
cd Driving-Sim
mkdir build
cd build
cmake ..
make
\end{lstlisting}

Une fois fait, il est possible de lancer les différents exécutables :
\begin{itemize}
    \item \texttt{Driving\_Sim} : pour le simulateur de conduite.
    \item \texttt{drivingSim\_test} : pour les tests unitaires.
    \item \texttt{drivingSim\_plot} : pour la génération des graphiques.
\end{itemize}

