\section{Bilan Technique}

\subsection{Problèmes rencontrés et surmontés}
Durant le développement du simulateur de conduite, plusieurs difficultés ont été rencontrées.
Tout d'abord, il a fallu se familiariser avec la bibliothèque SFML, qui était totalement inconnue pour nous, et imaginer une interface graphique adaptée.
De plus, l'utilisation du fichier \texttt{CMakeLists.txt} pour gérer les bibliothèques et la création des différents exécutables a représenté un défi technique supplémentaire.
Enfin, la gestion du temps s'est avérée délicate, notamment en raison de la charge liée aux autres projets universitaires en cours.

\subsection{Retour sur les technologies utilisées}
Le choix du langage C++ s'est révélé très judicieux du fait de sa polyvalence.
Bien que notre connaissance initiale du langage ait constitué une limite, ce projet nous a permis de progresser significativement dans sa maîtrise.
L'utilisation de SFML a également été un atout majeur, facilitant le développement d'applications graphiques sans nécessiter l'emploi de techniques excessivement complexes.

\subsection{Perspectives d'améliorations}
Pour les futures itérations du projet, plusieurs axes d'amélioration ont été identifiés :
\begin{itemize}
    \item Enrichir la diversité des circuits proposés pour varier l'expérience de conduite.
    \item Améliorer l'esthétique globale du projet, tant au niveau de l'interface graphique que des animations.
    \item Rendre le modèle physique encore plus réaliste en intégrant des phénomènes supplémentaires et en affinant les paramètres.
\end{itemize}

\subsection{Ce que nous ferions différemment}
Avec le recul, nous aurions intégré dès le départ des tests unitaires et des tests d'intégration afin de détecter plus tôt les erreurs et d'améliorer la robustesse du code.
Une phase de planification et de prototypage plus approfondie aurait permis de réduire les imprévus liés à l'apprentissage de nouvelles bibliothèques, notamment SFML.
Ce changement méthodologique aurait sans doute accéléré le développement et renforcé la qualité du produit final.

\subsection{Mise en route du projet}
\texttt{A rédiger: Instruction de compilation ...}