\usepackage{glossaries}\section{Sujet du projet}\label{sec:sujet-du-projet}
\subsection{Le sujet}\label{subsec:le-sujet}
Le sujet originel nous laissait une assez grande liberté quant à l'orientation de notre projet.
L'idée globale consistait en la réalisation graphique d'un mini-simulateur de conduite, avec des ordres de conduite (démarrage, arrêt, accélération et décélération) donnés par la frappe du clavier.
L'utilisateur devait pouvoir observer son véhicule ainsi que le circuit avec une vue du dessus ou la vision du conducteur à travers le pare-brise.
Il était aussi proposé de pouvoir \og corser\fg{} le jeu en programmant des événements aléatoires.

Pour réaliser ce projet, plusieurs langages de programmation nous étaient proposés :
\begin{itemize}
    \item \textbf{\gls{drracket}}, avec des compléments pour nous aider sur la partie graphique.
    \item \textbf{\gls{jruby}}, avec pour idée d'implémenter la logique du programme en \gls{ruby} et la partie graphique avec des bibliothèques de \gls{java}.
    \item \textbf{\gls{csharp}} ou \textbf{\gls{cpp}} avec les bibliothèques graphiques adéquates.
\end{itemize}


\subsection{Notre interprétation du sujet / Objectif du sujet}\label{subsec:notre-interpretation-du-sujet-/-objectif-du-sujet}
Après réflexion et discussion avec notre tuteur, nous avons décidé de partir vers une orientation didacticiel plutôt qu'une \og orientation de `jeu' \fg{}.
L'idée était d'implémenter un modèle de \gls{dynamique_vehicule} se rapprochant de la réalité.
Avec un tel modèle, nous pourrions donc visualiser des comportements de perte d'adhérence sur route, ainsi que les phénomènes de \gls{survirage} et \gls{sous_virage}.
Un tel simulateur permettrait ainsi à l'utilisateur d'expérimenter / se familiariser avec les comportements du véhicule.
Pour le langage de programmation, nous avons décidé de nous orienter vers le langage \gls{cpp}, car il nous était plus familier que les autres langages proposés, et nous avons utilisé \gls{sfml} en tant que bibliothèque graphique.
Ensuite, nous avons donc commencé l'implémentation de notre simulateur, il s'est décomposé en deux grandes parties, l'implémentation de la physique et la création de l'interface graphique.
Nous allons commencer par vous présenter la partie physique de notre projet.

\newpage